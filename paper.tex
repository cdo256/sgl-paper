\documentclass{article}
\usepackage{svg}
\usepackage{tikz}
\usepackage{authblk}
\usetikzlibrary{graphs}
\usetikzlibrary{arrows.meta}

\usepackage{cite}
\usepackage{tikz}
\usepackage{svg}
\usetikzlibrary{graphs}
\usetikzlibrary{arrows.meta}

\newcommand{\localGlyph}[1][]{%
  \includesvg[width=0.7em]{local-glyph.svg}
}

\pgfkeys{
  /pgf/arrow keys/.cd,
  glyph text/.code={%
    \pgfarrowsaddtooptions{%
      \def\cdo@glyph{\textsf{#1}}%
    }%
  },
  glyph command/.code={%
    \pgfarrowsaddtooptions{%
      \def\cdo@glyph{\csname #1\endcsname}%
    }%
  }
}

\pgfdeclarearrow{
  name=BigGlyph,
  cache=false,
  bending mode=none,
  drawing code={
    \pgfpathrectangle{\pgfpoint{-1.5ex}{-1ex}}{\pgfpoint{+2ex}{+2.5ex}}
    \pgfusepathqclip
    \pgftransformxshift{+0.1ex}
    \pgftransformyshift{-0.8ex}
    \pgftext[right,base]{$\cdo@glyph$}
  }
}

\pgfdeclarearrow{
  name=SmallGlyph,
  cache=false,
  bending mode=none,
  parameters={},
  drawing code={
    \pgfpathrectangle{\pgfpoint{-1.5ex}{-1ex}}{\pgfpoint{+2ex}{+2.5ex}}
    \pgfusepathqclip
    \pgftransformxshift{+0.1ex}
    \pgftransformyshift{-0.5ex}
    \pgftext[right,base]{$\cdo@glyph$}
  }
}

\pgfdeclarearrow{
    name = 0,
    means = {Arc Barb[arc=360]}
}

\pgfdeclarearrow{
    name = 1,
    parameters = { \the\pgfarrowlength },
    setup code = {
        \pgfarrowssettipend{0\pgfarrowlength}
        \pgfarrowssetlineend{0\pgfarrowlength}
        \pgfarrowssetvisualbackend{-1\pgfarrowlength}
        \pgfarrowssetbackend{-1\pgfarrowlength}
        \pgfarrowshullpoint{0\pgfarrowlength}{0pt}
        \pgfarrowshullpoint{-.5\pgfarrowlength}{.5\pgfarrowlength}
        \pgfarrowshullpoint{-1\pgfarrowlength}{0pt}
        \pgfarrowshullpoint{-.5\pgfarrowlength}{-.5\pgfarrowlength}
        \pgfarrowssavethe\pgfarrowlength
    },
    drawing code = {
        \pgfpathmoveto{\pgfqpoint{-.5\pgfarrowlength}{.5\pgfarrowlength}}
        \pgfpathlineto{\pgfqpoint{-.5\pgfarrowlength}{-.5\pgfarrowlength}}
        \pgfusepathqstroke
    },
    defaults = { length = 0.5em }
}

\pgfdeclarearrow{name=2,means={BigGlyph[glyph text=2]}}
\pgfdeclarearrow{name=3,means={BigGlyph[glyph text=3]}}
\pgfdeclarearrow{name=4,means={BigGlyph[glyph text=4]}}
\pgfdeclarearrow{name=5,means={BigGlyph[glyph text=5]}}
\pgfdeclarearrow{name=6,means={BigGlyph[glyph text=6]}}
\pgfdeclarearrow{name=7,means={BigGlyph[glyph text=7]}}
\pgfdeclarearrow{name=8,means={BigGlyph[glyph text=8]}}
\pgfdeclarearrow{name=9,means={BigGlyph[glyph text=9]}}

\pgfdeclarearrow{name=A,means={BigGlyph[glyph text=A]}}
\pgfdeclarearrow{name=B,means={BigGlyph[glyph text=B]}}
\pgfdeclarearrow{name=C,means={BigGlyph[glyph text=C]}}
\pgfdeclarearrow{name=D,means={BigGlyph[glyph text=D]}}
\pgfdeclarearrow{name=E,means={BigGlyph[glyph text=E]}}
\pgfdeclarearrow{name=F,means={BigGlyph[glyph text=F]}}
\pgfdeclarearrow{name=G,means={BigGlyph[glyph text=G]}}
\pgfdeclarearrow{name=H,means={BigGlyph[glyph text=H]}}
\pgfdeclarearrow{name=I,means={BigGlyph[glyph text=I]}}
\pgfdeclarearrow{name=J,means={BigGlyph[glyph text=J]}}
\pgfdeclarearrow{name=K,means={BigGlyph[glyph text=K]}}
\pgfdeclarearrow{name=L,means={BigGlyph[glyph text=L]}}
\pgfdeclarearrow{name=M,means={BigGlyph[glyph text=M]}}
\pgfdeclarearrow{name=N,means={BigGlyph[glyph text=N]}}
\pgfdeclarearrow{name=O,means={BigGlyph[glyph text=O]}}
\pgfdeclarearrow{name=P,means={BigGlyph[glyph text=P]}}
\pgfdeclarearrow{name=Q,means={BigGlyph[glyph text=Q]}}
\pgfdeclarearrow{name=R,means={BigGlyph[glyph text=R]}}
\pgfdeclarearrow{name=S,means={BigGlyph[glyph text=S]}}
\pgfdeclarearrow{name=T,means={BigGlyph[glyph text=T]}}
\pgfdeclarearrow{name=U,means={BigGlyph[glyph text=U]}}
\pgfdeclarearrow{name=V,means={BigGlyph[glyph text=V]}}
\pgfdeclarearrow{name=W,means={BigGlyph[glyph text=W]}}
\pgfdeclarearrow{name=X,means={BigGlyph[glyph text=X]}}
\pgfdeclarearrow{name=Y,means={BigGlyph[glyph text=Y]}}
\pgfdeclarearrow{name=Z,means={BigGlyph[glyph text=Z]}}

\pgfdeclarearrow{name=a,means={SmallGlyph[glyph text=a]}}
\pgfdeclarearrow{name=b,means={SmallGlyph[glyph text=b]}}
\pgfdeclarearrow{name=c,means={SmallGlyph[glyph text=c]}}
\pgfdeclarearrow{name=d,means={SmallGlyph[glyph text=d]}}
\pgfdeclarearrow{name=e,means={SmallGlyph[glyph text=e]}}
\pgfdeclarearrow{name=f,means={SmallGlyph[glyph text=f]}}
\pgfdeclarearrow{name=g,means={SmallGlyph[glyph text=g]}}
\pgfdeclarearrow{name=h,means={SmallGlyph[glyph text=h]}}
\pgfdeclarearrow{name=i,means={SmallGlyph[glyph text=i]}}
\pgfdeclarearrow{name=j,means={SmallGlyph[glyph text=j]}}
\pgfdeclarearrow{name=k,means={SmallGlyph[glyph text=k]}}
\pgfdeclarearrow{name=l,means={SmallGlyph[glyph text=l]}}
\pgfdeclarearrow{name=m,means={SmallGlyph[glyph text=m]}}
\pgfdeclarearrow{name=n,means={SmallGlyph[glyph text=n]}}
\pgfdeclarearrow{name=o,means={SmallGlyph[glyph text=o]}}
\pgfdeclarearrow{name=p,means={SmallGlyph[glyph text=p]}}
\pgfdeclarearrow{name=q,means={SmallGlyph[glyph text=q]}}
\pgfdeclarearrow{name=r,means={SmallGlyph[glyph text=r]}}
\pgfdeclarearrow{name=s,means={SmallGlyph[glyph text=s]}}
\pgfdeclarearrow{name=t,means={SmallGlyph[glyph text=t]}}
\pgfdeclarearrow{name=u,means={SmallGlyph[glyph text=u]}}
\pgfdeclarearrow{name=v,means={SmallGlyph[glyph text=v]}}
\pgfdeclarearrow{name=w,means={SmallGlyph[glyph text=w]}}
\pgfdeclarearrow{name=x,means={SmallGlyph[glyph text=x]}}
\pgfdeclarearrow{name=y,means={SmallGlyph[glyph text=y]}}
\pgfdeclarearrow{name=z,means={SmallGlyph[glyph text=z]}}


\title{SGL - A Natural Graphical Meta-Language and Rewriting System}
\author[1]{Christina O'Donnell }%\\ \href{mailto:cdo@mutix.org}{cdo@mutix.org}}
\affil[1]{Univerisity of Nottingham}
\begin{document}
\maketitle
\section{Introduction}
Programs such as proof assistants, SMT solvers and compilers that operate on
complex data-structures all require a representation of bound-variables. Many
attempts have been made to formalize the semantics of variable binding,
including category-theoretic models, Higher-Order Abstract Syntax (HOAS),
Nominal Techniques, De Bruijn Indices, first-order/nameful paradigms, and
graphical representations. 

\subsection{Desiderata}

Let's start by defining 'nice' properties that we want the machine to have. We pick features 

We want the following features:

\begin{itemize}
    \item No special constructs.
    \item Trivial capture-avoiding substitution. 
    \item Symmetry preserviing.
    \item Able to encode 'high-level' structures and algorithms.
    \item Relies on structure rather than name for variable binding.
    \item Cross-simulation.
    \item Cross-paradigm expressiveness.
\end{itemize}

\subsection{Other Approaches}
In representing abstract syntax with variable binding, several models and
schools of thought have been developed. This section reviews some of the
principal approaches in the literature.

\subsubsection{Syntax Graph Models}
Syntax graph models represent syntax structures as graphs, where nodes
correspond to operations or constructs and edges denote the relationships or
bindings between them. These models are particularly effective for visualizing
intricate dependencies and relationships in syntax.

\subsubsection{Category Theoretic Models}
Category theory offers a rigorous mathematical framework for modeling syntax
with variable binding. In this approach, categories of models correspond to
binding signatures, with each model comprising variable sets endowed with
algebraic and substitution structures. This method enables an abstract,
compositional treatment of syntax, supporting initial algebra semantics and
facilitating semantic substitution lemmas \cite{fiore99cat}.

\subsubsection{Higher-Order Abstract Syntax (HOAS)}
Higher-Order Abstract Syntax (HOAS) employs the binding mechanisms of the typed
lambda calculus to represent various forms of binding. This technique shifts the
complexity of variable binding to the meta-level, thereby ensuring uniform
handling of properties like renaming and substitution. Nevertheless, it
introduces challenges for structural recursion and induction \cite{gabbay02hoas}.

\subsubsection{Nominal Techniques}
Nominal techniques explicitly use names to handle variable binding, often
leveraging permutation models or nominal sets. This approach aids in reasoning
about alpha-equivalence and fresh name generation, making it especially suitable
for formal systems involving variable-binding operations \cite{pitts16nominal}.

\subsubsection{De Bruijn Indices}
De Bruijn indices provide a nameless representation of variables, using numeric
indices to signify variable scopes. This technique simplifies alpha-equivalence
handling, though it can make the syntax less intuitive to human readers
\cite{bruijn72lambda}.

Each of these approaches presents distinct advantages and limitations. The
choice of model often depends on specific application requirements, including
ease of reasoning, computational efficiency, and human readability.

\subsection{Contributions}
%TODO

\section{Defining Semantic Graph Language}

\subsection{Some new notation}

Semantic graphs are graphs like the below:

\begin{tikzpicture}
\node {+}
    child {node {3} edge from parent[-0]}
    child {node {2} edge from parent[-1]}
    ;
\end{tikzpicture}

\subsection{The predicate $L$}

\section{Graphical Pattern calculus}

\bibliography{paper}{}
\bibliographystyle{plain}
\end{document}
